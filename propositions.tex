%!TEX program = xelatex
%!TEX encoding = UTF-8 Unicode

\documentclass[parskip=half,
               fontsize=9pt]{scrartcl}

\usepackage{amsmath,amssymb}        % AMS symbols and environments
\usepackage{fontspec}               % Selecting fonts
\usepackage{unicode-math}           % Use unicode math font, not TeX
\usepackage[english]{babel}         % Correct hyphenation

% For printing on A4
\usepackage[a4paper,
            layoutwidth=17.955cm,
            layoutheight=23.351cm,
            layouthoffset=1.5225cm,
            layoutvoffset=3.1745cm,
            showcrop,
            top=2.170cm,
            bottom=3.510cm,
            inner=2.1835cm,
            outer=2.1835cm]{geometry}

% Actual size: 95% of crown quarto
% \usepackage[paperwidth=17.955cm,
%             paperheight=23.351cm,
%             top=2.170cm,
%             bottom=3.510cm,
%             inner=2.1835cm,
%             outer=2.1835cm]{geometry}

% Font setup
\setmainfont[Ligatures=TeX,ItalicFont=Feijoa-MediumItalic,StylisticSet=6]{Feijoa}
\setmonofont[BoldFont=GTPressuraMono-Bold,ItalicFont=GTPressuraMono-LightItalic]{GTPressuraMono-Light}
\setmathfont{Asana-Math.otf}
\newfontfamily\fanciestfont[Ligatures={TeX,Discretionary}]{Feijoa-Display}
\newfontfamily\fancyfont[Ligatures=TeX]{Feijoa-Display}
\newfontfamily\chapternumberfont[Ligatures=TeX,Numbers=Lining]{Feijoa-Display}

\begin{document}

\pagestyle{empty}

\begin{center}

Propositions accompanying the thesis

{\Large\fanciestfont{}Higher-dimensional modelling of geographic information}

by

{\Large Ken Arroyo Ohori}

\end{center}

\bigskip

\begin{enumerate}

% About thesis

\item
Using higher-dimensional topological data structures, many difficult problems can be flattened to equivalent---but conceptually simpler---problems on graphs. [\emph{This thesis}]

\item
Two representations will come to dominate geographic information of any dimension: point clouds and space subdivisions. [\emph{This thesis}]
% Representing roads as lines and buildings as unconstrained soups of faces will soon seem as anachronistic as wireframe models. 

% Side topics / GIS in general

\item
Computational methods will replace geoscientific methods in almost every GIS application:
\emph{fast reasonable approximations} on large noisy datasets will increasingly trump \emph{slow optimal solutions} on a few carefully acquired data points.

\item
A lack of concern for geometric and topological correctness is the main reason behind the lack of successful applications for general-purpose 3D models.

\item
In computer science, it is almost always better to speak of research objectives than research questions.

% Science & thesis

\item
There is no such thing as a predatory publisher, only predatory metrics used to gauge the value of a scientist.

\item
In the future, peer review will be conducted post publication.

\item
Good academic writing advice encourages clarity and expressiveness as much as it discourages formulaic constructions.

% Politics

\item
Innovation in internet services will occur in countries that enshrine net neutrality in law.

\item
The results of research performed at public universities or using public funds ought to be released into the public domain.

\end{enumerate}

\vfill
These propositions are regarded as opposable and defendable, and have been approved as such by the promotor Prof.\ dr.\ J.\ Stoter and the copromotor Dr.\ H.\ Ledoux.

\end{document}